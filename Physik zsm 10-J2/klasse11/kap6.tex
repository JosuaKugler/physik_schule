\section{Interferenzphänomene}
		\subsection{Interferenz mit zwei Quellen}
			\paragraph{a) Wichtige Voraussetzungen für Interferenzphänomene:} 
			Zwei Sender mit gleicher Frequenz und konstanter Phasendifferenz nennt man kohärent. Kohärenz ist eine unabdingbare Voraussetzung für die Entstehung eines über längere Zeit beobachtbaren Interferenzphänomens.
			
			\paragraph{b)} Ein beliebiger Oszillator im Wellenfeld wird immer von zwei Wellen erfasst und von beiden zu erzwungenen Schwingungen der selben Frequenz angeregt. Entscheidend für das Ergebnis der Überlagerung in einem Punkt ist die dortige Phasendifferenz $\Delta \varphi$ der resultierenden Schwingungen. Diese ergibt sich aus dem Gangunterschied $\Delta$.
			
			\paragraph{Konstruktive Interferenz} $ \delta = k \ast \lambda $ ; $k \in \mathbb{N}_0$ \\
			\hspace{40mm} $ \Delta \varphi = k \ast 2\pi$
			
			\paragraph{Destruktive Interferenz} $ \delta = (2k-1) \ast \frac{\lambda}{2} $ ; $k \in \mathbb{N} $ \\
			\hspace{40mm} $ \Delta \varphi = (2k-1) \ast \pi $ 
			\vspace{5mm} \\
			$ \frac{\delta}{\lambda} \ast 2\pi = \Delta \varphi$
			
			\paragraph{Interferenzkurven:} Zwei kohärente Kreiswellen erzeugen durch Interferenz ein symmetrisches Wellenfeld aus konfokalen Interferenzhyperbeln konstruktiver und destruktiver Interferenz. Die Punkte mit Gangunterschied $ \delta = k \ast \lambda $ $(k \in \mathbb{N}) $ liegen auf Hyperbeln konstruktiver Interferenz, die Punkte mit Gangunterschied $ \delta = (2k-1) \ast \frac{\lambda}{2}$ $(k \in \mathbb{N}) $ auf Hyperbeln destruktiver Interferenz. 
			
			\paragraph{Energieverteilung: } Die Energie an einem Ort ist proportional zu $\hat{s}^{\hspace{1mm} 2}$. An Orten maximaler Amplitude ist $ W_{ges} \sim (2 \ast \hat{s})^{2} = 4 \ast \hat{s}^{\hspace{1mm} 2}$, an Orten destruktiver Interferenz ist $W_{ges} = 0 $. Im Mittel ergibt sich also $ W_{ges} \sim 2 \ast \hat{s}^{\hspace{1mm} 2}$. Dies ist das selbe Ergebnis, das man durch die Addition der Schwingung zweier fortschreitender Wellen erhält. 
			\vspace{2mm} \\
			Durch Interferenz wird die Energieverteilung im Feld geändert, die Energiesumme bleibt jedoch erhalten. 
			
			
			\newpage\noindent
			Interferenzmaximum: Zeiger zeigen in die gleiche Richtung / Phasendifferenz 0 \\
			Interferenzminimum: Zeiger zeigen in gegensätzliche Richtung / Phasendifferenz Pi
			\vspace{10mm}\\
			
		\subsection{Berechnung der Lage der Maxima und Minima im Interferenzfeld mithilfe trigometrischer Beziehungen}
		
		Die Wellenstrahlen $s_1$ und $s_2$ verlaufen fast parallel unter dem Winkel $\alpha$ zur Mittelsenkrechten (optische Achse) und treffen sich im Punkt P des um a entfernten Schirms im Abstand d von der Mitte (des Schirms). 
		\vspace{5mm} \\
		Es gilt: 
		\vspace{2mm} \\
		$ tan (\alpha) = \frac{d}{a} $
		\vspace{2mm} \\
		$ sin (\alpha) = \frac{\delta}{g} $
		
		\subsection{Das Huygens'sche Prinzip (Nach Christiaan Huygens, 1629-1695)}
		
			\subsubsection{MERKE} 
			
			Jeder Punkt einer Wellenfront kann als Ausgangspunkt von Elementarwellen angesehen werden (Mit gleichem $\Delta \varphi$ und f wie die ursprüngliche Welle). Die einhüllende aller Elementarwellen ergibt die neue Wellenfront. (Skizze1)
			
			\subsubsection{Brechung und Reflexion erklärt durch das Huygens'sche Prinzip:} 
			
			Beim Übergang von einem optisch dichteren Stoff in einen optisch dünneren Stoff kommt es zur Brechung. Der Zusammenhang zwischen den Brechungswinkeln und den Ausbreitungsgeschwindigkeiten des Lichtes in den beiden Stoffen kann man mit dem Huygens'schen Prinzip herleiten.  Es gilt: 
			\vspace{2mm} \\
			$ \frac{sin(\alpha)}{sin(\beta)} = \frac{c_1}{c_2} = n $ ; Brechzahl $ n>1 $
			\vspace{5mm} \\
			Falls $ \alpha = 90^{\circ} $ ergibt sich:
			\vspace{1mm} \\
			$ \frac{1}{sin(\beta_{grenz})} = n$
			\vspace{5mm} \\
			Erinnerung: Der Lichtweg ist umkehrbar. Wenn Licht aus dem dichteren Medium unter einem größeren Winkel als $\beta_{grenz}$ auf die Oberfläche trifft, wird es total reflektiert.
			\newpage
			
			\subsubsection{Beugung von Wellen erklärt durch das Huygens'sche Prinzip}
			
			Das eindringen von Wellen in den geometrischen Schattenraum hinter Hindernissen oder Öffnungen wird als Beugung bezeichnet. 
			
			Bei Doppelspaltversuchen erzeugt man aus der Welle eines Senders durch den Doppelspalt zwei gleichphasige Elementarwellen, die genauso miteinander interferieren wie die Wellen von zwei kohärenten Erregerzentren.