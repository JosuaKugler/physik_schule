\section{Schwingungskreis}
\subsection{Wdh. Kondensator}
\subsection{EES im Schwingkreis}
$W_{ges}$ ist zeitlich konstant:
\begin{align*}
	\dfrac{1}{2}L*I(t)^2+\dfrac{1}{2}*C*U(t)^2&=konst&&|()'\\
	L*I(t)*I'(t)+C*U(t)*U'(t)&=0&&|U=-L*I'\\
	L*I(t)*I'(t)+C*(-L*I'(t))*U'(t)&=0\\
	I'(t)*L*\left(I(t)-C*\dfrac{Q'(t)}{C}\right)&=0
\end{align*}
\subsection{Vgl. von Schwingkreis mit mech. Schwingung}
\begin{align*}
	U_L&=U_C\\
	-L*I'(t)&=\dfrac{Q(t)}{C}\\
	-L*Q''(t)&=\dfrac{1}{C}*Q(t)\\
	L*Q''(t)+\dfrac{1}{C}*Q(t)&=0&&|\mathrm{DGL\ des\ Schwingkreis}\\
	m*s''(t)+D*s(t)&=0&&|\mathrm{DGL\ der\ mech.\ Schwingung}\\
\end{align*}
\begin{tabular}{l|l}
	SK & MSchw.\\
	\hline $L$&$m$\\
	$\frac{1}{C}$&$D$\\
	$Q$&$s$\\
	$Q'=I$&$v=s'$\\
	$I'$&$a=s''$\\
	$-\frac{1}{C}*Q=-U_C$&$-D*s(t)=F_R$\\
	$-L*I'(t)$ &$-m*a(t)$\\
	$\frac{1}{2}*L*I^2$&$\frac{1}{2}*m*v^2$\\
	$\frac{1}{2}*\frac{1}{C}*Q(t)^2=\frac{1}{2}*C*U(t)^2$&$\frac{1}{2}*D*s^2$	
\end{tabular}

\subsection{Wdh. Energieerhaltungssatz der Mechanik}
\begin{align}
	F(t)&=m*a(t)&&|*v(t)\\
	F(t)*v(t)&=m*a(t)*v(t)\\
	F(t)*v(t)&=m*v'(t)*v(t)&&|\int \mathrm{d}t\\
	\int F(t)*s'(t)\mathrm{d}t&=\int m*v'(t)*v(t)\mathrm{d}t\\
	-\int V'(s(t))*s'(t)\mathrm{d}t&=\int m*v'(t)*v(t)\mathrm{d}t\\
	-V'(s(t))+c_1&=\dfrac{1}{2}*m*v(t)^2+c_2\\
	c&=\dfrac{1}{2}*m*v(t)^2+V'(s(t))\\
	const&=E_{kin}+E_{pot}
\end{align}
Der Energieerhaltungssatz ist nur dann erfüllt, wenn es eine Funktion gibt, für die gilt: \[-V'(s)=F\]

\subsection{Maxwellgleichungen}
\begin{subequations}
	\begin{align}
	\underbrace{\vec{\nabla}}_{\mathrm{Quelle\ von}}\ast\vec{E}=\dfrac{\sigma}{\epsilon_0}&&|\vec{\nabla}=\begin{pmatrix}\partial x\\\partial y\\\partial z\end{pmatrix}\\
	\vec{\nabla} \ast\vec{B}=0\\
	\underbrace{\vec{\nabla} \times}_{\mathrm{Wirbel\ von}} \vec{E}=-\dfrac{\partial B}{\partial t}&&|\mathrm{ Minus\ wegen\ Lenz}\\
	\vec{\nabla} * \vec{B}=\mu_0*J+\mu_0*\epsilon_0*\dfrac{\partial E}{\partial t}
	\end{align}
\end{subequations}
Im Vakuum sind (a) und (b) symmetrisch. (c) und (d) sind ebenfalls symmetrisch.
\begin{subequations}
	\begin{align}
		\vec{\nabla}\ast\vec{E}&=\dfrac{\sigma_e}{\epsilon_0}\\
		\vec{\nabla}\ast\vec{B}&=\mu_0*\sigma_m\\
		\vec{\nabla}\times \vec{E}=-\mu_0*J_m-\dfrac{\partial B}{\partial t}\\
		\vec{\nabla} \times \vec{B}=\mu_0*J_e+\mu_0\epsilon_0\dfrac{\partial E}{\partial t}	
	\end{align}
\end{subequations}


