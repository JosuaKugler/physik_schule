\documentclass[12pt]{article}
\title{Physik Klasse 10}
\author{Andreas L\"orincz, Saskia Grasl, Lucca K\"ummerle}
\usepackage{ucs}
\usepackage[utf8x]{inputenc}
\usepackage[T1]{fontenc}
\usepackage[ngerman]{babel}
\usepackage{amsmath,amssymb,amstext}
\usepackage[fleqn,tbtags]{mathtools}
\numberwithin{equation}{subsection}
\begin{document}
	\maketitle
	\tableofcontents
	
	\section{Mechanik (Wiederholung)}
	Die Mechanik befasst sich mit der Bewegung von Körpern, den auf diese wirkenden Kräfte und mit den zugehörigen Größen wie zum Beispiel der Energie.
	
	\subsection{Die gleichförmige Bewegung}
	Die Geschwindigkeit $ v $ ist bei der gleichförmigen Bewegung fest. Die Kurve im $ t $-$ s $-Diagramm ist eine lineare Funktion.
	% Diagramm
	Wichtig für später: Im $ t $-$ v $-Diagramm findet man die zurückgelegte Strecke $ x $ als Fläche unter dem Graphen wieder.
	Im $ t $-$ x $-Diagramm ist die Geschwindigkeit $ v $ in der Steigung (mathematisch: Ableitung) des Graphen wiederzufinden.
	
	\subsection{Der Impuls $p$}
	\paragraph{Versuch}
	% Bild
	Durch die (mehrfache) heroische Arbeit unseres Laborassistenten konnten wir folgende Beziehung zwischen Impuls $ p $, Masse $ m $ und Geschwindigkeit $ v $ nachweisen:
	\begin{itemize}
		\item Je höher die Masse $ m $ eines Körpers, desto höher der Impuls $ p $ ($ v $ konstant!).
		\item Je höher die Geschwindigkeit $ v $ eines Körpers, desto höher der Impuls $ p $ ($ m $ konstant!).
	\end{itemize}
	\paragraph{Zusammengefasst:}
	\begin{equation}
		p=m*v
	\end{equation}
	\paragraph{Weitere Eigenschaften:}
	\begin{itemize}
		\item Der Impuls $ p $ ist wie auch die Geschwindigkeit $ v $ und später auch die Kraft, eine Vektorgröße, d.h. er besitzt eine Richtung. Wenn dies von Bedeutung ist, wird zur Kenntlichmachung ein Vektorpfeil (Pfeil nach rechts) über dem Buchstaben gezeichnet.
		\item Der Impuls $ p $ ist eine Erhaltungsgröße, d.h. in einem abgeschlossenen System ist die Summe aller Impulse konstant.
	\end{itemize}
	
	\subsection{Die Kraft $F$}
	Um den Impuls $ p $ eines Körpers zu ändern, wird eine Kraft $ F $ benötigt (der konkrete mathematische Zusammenhang wird später behandelt).
	\paragraph{Beispiel:}
	Ein Spielzeugauto wird in einer realistischen Umgebung beschleunigt und dann losgelassen.
	% Bild
	Durch die Reibung zwischen Auto und Boden liegt eine Kraft $ F_{Reib} $ vor, die den Impuls des Autos abbaut, bis das Auto stoppt.
	\paragraph{Beispiel 2:}
	% Bild
	Auf Andreas wirken (im Gegensatz zu Benjamin) zwei Kräfte:\\
	Neben der Gravitationskraft $ F_G $ die ihn in Richtung Erdmittelpunkt beschleunigt, übt der Boden eine Gegenkraft $ F_{Gegen} $ mit gleicher Stärke entgegen der Gravitationskraft aus.
	\subparagraph{Folge:}
	Andreas behält seine Position bei, während er Benjamin zusieht, wie dieser nach unten beschleunigt wird.\\\\
	Wenn sich alle auf einen Körper wirkende Kräfte aufheben, dann liegt ein sogenanntes \underline{Kräftegleichgewicht} vor.
	
	\section{Kräfte im Raum}
	\paragraph{Wiederholung:}
	Kräfte werden durch 3 Eigenschaften charakterisiert:
	\begin{itemize}
		\item Richtung
		\item Betrag
		\item Angriffspunkt
	\end{itemize}
	Wir wissen schon, wie zwei oder mehr Kräfte miteinander verrechnet werden müssen, wenn sie (anti-)parallel zueinander sind. Im folgenden soll die Kombination von nicht parallelen Kräften untersucht werden.
	
	\subsection{Kräfteaddition}
	\paragraph{Aufgabenstellung:}
	Gegben seien zwei Kräfte $ \vec{F_1} $ und $ \vec{F_2} $. Wie bestimmt man die daraus resultierende Gesamtkraft $ \vec{F_{Res}} $? Praktikum s. AB.
	\paragraph{Feststellung:}
	Die beiden Kräfte $ \vec{F_1} $ und $ \vec{F_2} $ bilden die Seiten eines Parallelogramms, während die resultierende Kraft $ \vec{F_{Res}} $ die Diagonale dieses Parallelogramms ist.\\\\
	Um zeichnerisch die Addition von zwei Kräften durchzuführen gibt es zwei Methoden:
	\paragraph{Erste Methode:}
	\begin{enumerate}
		% Bild, Text einfärben
		\item Zeichne die beiden Kraftpfeile $ \vec{F_1} $ und $ \vec{F_2} $ ein
		\item Zeichne eine parallele Gerade zu einem der Pfeile, der dann durch die Spitze des anderen Pfeils geht (bei beiden Pfeilen so machen)
		\item Der Schnittpunkt ist der Endpunkt des resultierenden Gesamtkraftpfeils $ \vec{F_{Res}} $
		\item Benutzte Pfeile wegstreichen
	\end{enumerate}
	\paragraph{Zweite Methode:}
	\begin{enumerate}
		%Bild, Text einfärben
		\item Zeichne die beiden Kraftpfeile $ \vec{F_1} $ und $ \vec{F_2} $ ein
		\item Zeichne eine parallele Gerade zu einem der Pfeile, der dann durch die Spitze des anderen Pfeils geht
		\item Miss die Länge von $ \vec{F_2} $ aus
		\item Trage die gemessene Strecke von der Spitze von $ \vec{F_1} $ ausgehend auf der Parallelen auf
		\item Der Endpunkt dieser Strecke ist die Spitze von $ \vec{F_{Res}} $
		\item Benutzte Pfeile wegstreichen
	\end{enumerate}
	
	\subsection{Die Kraftzerlegung}
	In diesem Kapitel geht es um die Zerlegung einer Kraft in zwei Kräfte entlang relevanter Richtungen
	\paragraph{Versuch:}
	%Bild
	Ziel des Versuches war, das Seil zwischen den Personen gerade zu strecken. Dies gelang fast (nur ein 10°-Knick).\\\\
	Erläuterungen der Verhältnisse mit Kraftpfeilen. Gravitationskraft:
	\begin{equation}
		F_G=m*g=5kg*10\dfrac{N}{kg}=50N
	\end{equation}
	Eine gleich große Kraft $ F_{Gegen} $ wird in entgegengesetzte Richtung aufgebracht. Diese Kraft wird jedoch nicht direkt aufgebracht, sondern ist das Resultat der beiden Zugkräfte von Person A und B. Diese Kräfte verlaufen entlang der gespannten Seile.\\
	Hilfskonstruktion: Gerade auf Seile einzeichnen. Dies sind sogenannte \underline{Wirkungslinien}.\\\\
	Um das Kräfteparallelogramm zu konstruieren, zeichne zwei Parallelen zu den Wirkungslinien, die durch die Spitze von $ F_{Gegen} $ gehen. Die zwei Kräfte entlang der Wirkungslinien gehen vom Angriffspunkt aus hin zu den Schnittpunkten aus Wirkungslinien und Parallelen.\\
	Für die Belastungskraftaufsplittung macht man das selbe, nur eben nach unten.
	
	\paragraph{Allgemein:}
	Konstruktion bei gegebenem Kraftpfeil $ \vec{F} $, der auf zwei Abschnitte an einem Angriffspunkt wirkt:
	% Bild/Skizze
	\begin{enumerate}
		\item Zeichne die Wirkungslinien entlang der Abschnitte ein
		\item Zeichne Parallelen zu den Wirkungslinien durch die Spitze des Kraftpfeils $ \vec{F} $
		\item Kraftkomponenten $ \vec{F_1} $ und $ \vec{F_2} $ verlaufen vom Angriffspunkt aus hin zu den Schnittpunkten aus Wirkungslinien und Parallelen
	\end{enumerate}
	\paragraph{Zurück zum Einstiegsbeispiel:}
	% Bild/Skizze
	\subparagraph{Betrachtung:}
	Das Kräfteparallelogramm ist (wenn $ \vec{F_1} $ und $ \vec{F_2} $ gleichgroß sind) nicht nur ein Parallelogramm, sondern auch eine Raute.
	% Bild/Skizze
	\begin{subequations}
		\begin{align}
			\beta &= \dfrac{\alpha}{2}\\
			\sin\beta &= \dfrac{\dfrac{1}{2}\vec{F_{Res}}}{\vec{F_2}}\\
			\vec{F_2} &= \dfrac{\dfrac{1}{2}\vec{F_{Res}}}{\sin\beta}\\
			& \approx 278N
		\end{align}
	\end{subequations}
	
	\subsubsection{Kräfte an einem Hang}
	Die Kraftzerlegung eignet sich auch dazu, beispielsweise die Gravitationskraft, die auf einen Körper wirkt an einem, Hang in bewegungsrelevante Kräfte zu zerteilen.
	% Bild/Skizze
	\begin{description}
		\item[$ F_H $] ist die sogenannte Hangabtriebskraft und beschreibt die beschleunigende Kraft, die auf den Körper entlang des Hangs wirkt.
		\item[$ F_N $] ist die Normkraft und gibt an, wie stark der Körper auf die Oberfläche des Hangs drückt.
	\end{description}
	-->AB
	\paragraph{Es gilt also:}
	% Bild/Skizze
	\begin{equation}
		\dfrac{F_H}{F_G} = \dfrac{h}{s} = \sin\alpha
	\end{equation}
	Demzufolge kann man die Hangabtriebskraft aus der Gravitationskraft und dem Hangwinkel wie folgt berechnen:
	\begin{equation}
	F_H = F_G*\sin\alpha
	\end{equation}
	Analog folgt für die zu $ F_H $ senkrecht stehende Normkraft $ F_N $:
	\begin{equation}
	F_N = F_G*\cos\alpha
	\end{equation}
	$ F_N $ ist mit einem Faktor ($ \mu_t $) mit der Reibung verbunden ($ t $ ist entweder $ Haft $, $ Gleit $ oder $ Roll $). Die Haftreibung ist beispielsweise:
	\begin{align}
		F_{Haft} &= \mu_{Haft}*F_N\\
		&= \mu_{Haft}*F_G*\cos\alpha
	\end{align}
	Der Haftreibungskoeffizient ist abhängig davon, welche Materialien miteinander in Kontakt stehen.
	% Bild/Skizze
	Um den Haftreibungskoeffizienten $ \mu_{Haft} $ in dem praxisnahen Fall Taschenrechner-Tagebuch zu untersuchen wird das Tagebuch immer stärker geneigt, bis der Taschenrechner auf dem Tagebuch herunterrutscht. Unter de, gemessenen Winkel ist dann die \underline{Hangabtriebskraft} $ F_H $ größer als die \underline{Reibungskraft} $ F_{Reib} $.\\\\
	\emph{Messung:} $ \alpha = 27° $
	\begin{align}
		F_H &\geqq F_{Reib}\\
		F_G*\sin\alpha &\geqq \mu_{Haft}*F_G*\cos\alpha\ && |:(F_G*\cos\alpha)\\
		\dfrac{F_G*\sin\alpha}{F_G*\cos\alpha} &\geqq \mu_{Haft}\\
		\tan\alpha &\geqq \mu_{Haft}\\
		\nonumber\\
		\Rightarrow\hspace{0.5cm} \mu_{Haft} &\leqq 0,51 \quad \text{(Einheitenlos)}
	\end{align}
	
	\subsection{Die gleichmäßig beschleunigte Bewegung}
	\paragraph{Alt:}
	\begin{itemize}
		\item Gleichförmige Bewegung
		\item Geschwindigkeit $ v $ ist konstant
		\item $ t $-$ s $-Diagramm ist eine lineare Funktion
	\end{itemize}
	\paragraph{Neu:}
	\begin{itemize}
		\item Gleichmäßig beschleunigte Bewegung
		\item Geschwindigkeit $ v $ nimmt konstant zu
		\item $ t $-$ v $-Diagramm ist eine lineare Funktion
	\end{itemize}

	\subsection{Die Newtonschen Gesetze}
	\subsubsection{Erstes Newtonsches Gesetz}
	\paragraph{Gedankenexperiment:}
	% Bild
	Die Voyager-Sonde fliegt durch das Weltall. Sei die Geschwindigkeit zum Zeitpunkt $ t=0s $ $ v=500\frac{m}{s} $. Auf die Sonde wirkt keine Kraft, wie groß ist ihre Geschwindigkeit bei $ t=5s $?
	\subparagraph{Antwort:}
	Noch immer $ v=500\frac{m}{s} $, da keine beschleunigende Kraft wirkt.
	\paragraph{Erstes Newtonsches Gesetz:}
	Wirkt auf einen Körper keine Kraft, so bleibt sein Bewegungszustand (und somit sein Impuls) erhalten.
	
	\subsubsection{Zweites Newtonsches Gesetz}
	\paragraph{Gedankenexperiment 2:}
	% Bild
	Überlegungen zum Bremsvorganf des Einkaufswagens: Es hängen folgende Größen miteinander zusammen: Kraft $ F $, Beschleunigung $ a $ und Masse $ m $.\\
	Vermutungen:
	\begin{itemize}
		\item bei konstanter Masse $ m $: Je größer $ a $ sein soll, desto größer muss $ F $ sein
		\item bei konstanter Kraft $ F $: Je größer $ m $, desto kleiner $ a $
		\item bei konstanter Beschleunigung $ a $: Je größer $ m $, desto größer $ F $
	\end{itemize}
	Bei der dritten Versuchsreihe vermuten wir eine proportionale Beziehung zwischen beschleunigender Kraft $ F $ und Gesamtmasse $ m $, d.h. dass, wenn der Quotient aus Kraft und Gesamtmasse konstant ist ($ \frac{F}{m} $), die Beschleunigung auch konstant bleibt. Dies wurde durch die Untersuchung qualitativ bestätigt.
	\paragraph{Zusammenfassung:}
	\begin{align}
		\left.
		\begin{array}{c}
		F \sim a \quad\text{(m const.)}\\
		\\
		F \sim m \; \quad\text{(a const.)}
		\end{array}
		\right\}
		F \sim m*a
	\end{align}
	Um eine Formel daraus zu machen fehlt eine Proportionalitätskonstante. Nennen wir sie $c$. Die Formel lautet dann also:
	\begin{align}
		F &= c*m*a\\
		c &= \dfrac{F}{m*a}
	\end{align}
	Bestimmung von $ c $ mithilfe der Tabelle (mit $ F $ in $ N $, $ m $ in $ kg $ und $ a $ in $ \frac{m}{s^2} $)
	\begin{equation}
		c = \dfrac{0,02N}{0,206kg*0,09\dfrac{m}{s^2}} \approx 1,08
	\end{equation}
	Bei einer genauen Untersuchung (mit genau bestimmtem $ g $) ergibt sich $ c=1 $. Die Grundgleichung der Mechanik (=Newtons zweites Gesetz) lautet also:
	\begin{equation}
		F = m*a
	\end{equation}
	
	\subsubsection{Drittes Newtonsches Gesetz}
	\paragraph{Versuch:}
	% Bild
	\subparagraph{Beobachtung:}
	Wirft Person $ A $ den Ball $ B $ nach rechts, dann wird Person $ A $ nach links beschleunigt.
	\subparagraph{Folgerung:}
	Auch auf Person $ A $ wird beim Wurf eine Kraft in entgegengesetzte Richtung aufgebracht. Dies ist das dritte Newtonsche Gesetz.
	\paragraph{Drittes Newtonsches Gesetz:}
	Übt ein Körper $ A $ eine Kraft $ \vec{F_{AB}} $ auf einen Körper $ B $ aus, so erfährt Körper A ebenfalls eine Kraft $ \vec{F_{BA}} $ gleicher Größe, aber entgegengesetzter Richtung. Es gilt also:
	\begin{align}
		\mid\vec{F_{AB}}\mid &= \ \ \mid\vec{F_{BA}}\mid\\
		\vec{F_{AB}} &= -\vec{F_{BA}}
	\end{align}
	Wird eine Kraft $ F_{AB} $ auf einen Körper B angewandt und dauert dies die Zeit $ \Delta t $, dann folgt:
	\begin{align}
		F_{AB}*\Delta t &= m_B*a*\Delta t\\
		&= m_B*\Delta v\\
		&= \Delta p_B
	\end{align}
	Dies ist also die Impulsänderung. Wenn man Newton 3 analog für die andere Seite anwendet, erhält man:
	\begin{equation}
		\Delta p_B = -\Delta p_A
	\end{equation}
	Wenn man beide Impulsänderungen $ \Delta p_A $ und $ \Delta p_B $ addiert (also alle Impulsänderungen zusammenfasst), dann kommt heraus:
	\begin{equation}
		\Delta p_B+\Delta p_A = 0
	\end{equation}
	D.h. der Gesamtimpuls dieses abgeschlossenen Systems bleibt erhalten.
	
	\subsection{Die überlagerte Bewegung}
	Aus dem Übungsblatt ``Übungen zur Grundgleichung der Mechanik'' haben wir in f) erarbeitet, dass wenn eine Bewegung die Summe aus verschiedenen Bewegungsarten ist (im Beispiel: gleichförmige und gleichmäßig beschleunigte Bewegung), man die Bewegungsarten separat berechnen kann und anschließend erst zusammenaddieren kann.
	\paragraph{Teil 2:}
	Bewegungen müssen nicht nur entlang einer Raumachse stattfinden (z.B. schräger Wurf). Wenn die Bewegung in eine Raumrichtung die Bewegung in eine andere Raumrichtung nicht beeinflusst, dann kann man die Ortsverläufe in den einzelnen Raumrichtungen zu verschiedenen Zeitpunkten $ t $ separat berechnen und die errechneten Koordinaten in einer Ortsverlaufskurve zusammenfassen.
	
	\subsection{Erhaltungssätze}
	\subsubsection{Der Energieerhaltungssatz}
	Aus den Hebelgesetzen bzw. Versuchen mit dem Flaschenzug haben wir herausgefunden:
	% Bild
	\begin{align}
		\intertext{Was man an Weg spart, muss man an Kraft ausgleichen. Es gilt also:}
		F_1*s_1=F_2*s_2
	\end{align}
	Das Produkt aus beiden ist erhalten. Hinter diesem Produkt steht als Größe die \underline{Energie} $ E $.\\\\
	\emph{DIE ENERGIE IST ERHALTEN}\\\\
	Wird die Masse hochgezogen, dann wird die Energie, die die Masse gewinnt, als Höhenenergie bzw. potentielle Energie $ E_{pot} $ bezeichnet.
	\begin{equation}
		E_{pot} = F*s = F_G*h = m*g*h
	\end{equation}
	
	\subsubsection{Die kinetische Energie}
	\paragraph{Versuch:}
	Eine Kugel rollt einen Hang hinab. Sie verliert an potentieller Energie. Die Gesamtenergie eines Systems ist aber konstant.
	\paragraph{Folgerung:}
	Die potentielle Energie wurde in eine andere Energie umgewandelt, nämlich in Bewegungsenergie $ E_{kin} $ (auch kinetische Energie genannt).
	% Bild/Skizze
	\paragraph{Überlegung:}
	Je höher die Geschwindigkeit $ v $, desto größer die zugehörige kinetische Energie $ E_{kin} $.
	% (Bild)
	\paragraph{Gedankenexperiment:} Eine Kugel wird aus der Höhe $ h $ auf den Boden fallen gelassen. Die Höhenenergie $ E_{pot} $, die umgewandelt ist, beträgt:
	\begin{equation}
		E_{pot} = m*g*h
	\end{equation}
	Die Geschwindigkeit $ v $, die die Kugel unten ankommend hat, beträgt mithilfe der Formel für gleichmäßig beschleunigte Bewegung:
	\begin{align}
		h = s &= \dfrac{1}{2}*a*t^2\\
		&= \dfrac{1}{2}*\dfrac{a*t^2*a}{a}\\
		&= \dfrac{1}{2}*\dfrac{a^2*t^2}{a} && |v=a*t\\
		&= \dfrac{1}{2}*\dfrac{v^2}{a}\\
		v &= \sqrt{2*a*s}
	\end{align}
	Ist die Kugel unten angekommen, dann wurde die gesamte potentielle Energie in kinetische Energie umgewandelt:
	\begin{align}
		E_{pot} &= E_{kin}\\
		m*g*h &\stackrel{(\wedge)}{=} m*g*\dfrac{1}{2}*\dfrac{v^2}{a}\\
		&\stackrel{g=a}{=} m*\dfrac{1}{2}*v^2\\
		\nonumber\\
		\Rightarrow\hspace{0.5cm} E_{kin} &= \dfrac{1}{2}*m*v^2
	\end{align}
	
	\subsubsection{Die Spannenergie}
	Versuch mit Federn mit bestimmten Eigenschaften (Hookesche Federn)... s. Praktikum.
	\paragraph{Ergebnis:}
	$ F $ ist proportional zur Streckung $ s $. Der Proportionalitätsfaktor ist die sogenannte Federhärte $ D $.
	% Diagramm
	\begin{align}
		F &= D*s\\
		\nonumber\\
		[D] = \left[\dfrac{F}{s}\right] &= \dfrac{1N}{1m}
	\end{align}
	\paragraph{Vergleich:}
	Ein Massestück wird von $ h=0 $ bis $ h=h_1 $ angehoben. Wie hängen $ F $, $ \Delta h $ und $ E $ zusammen?
	% Diagramm
	\begin{align}
		E &= F*\Delta h\\
		&= m*g*\Delta h
	\end{align}
	$ E $ ist die Fläche unter der Kurve.
	\paragraph{Jetzt:}
	Wo verbirgt sich die Spannenergie $ E_{spann} $ im $ s $-$ F $-Diagramm der Feder?
	% Diagramm
	\begin{align}
		E_{spann} &= \dfrac{1}{2}*F*s\\
		E_{spann} &= \dfrac{1}{2}*D*s^2
	\end{align}
	
	\subsection{Die Impulserhaltung}
	\subsubsection{Der vollständig inelastische Stoß}
	Beim vollständig inelastischen Stoß kollidieren zwei Körper miteinander und bewegen sich nach dem Stoß gemeinsam mit der gleichen Geschwindigkeit weiter.
	\paragraph{Zu erwartende Formeln:}
	\begin{align}
		\shortintertext{Impuls:}
		p_1 + p_2 &= p'\\
		m_1*v_1+m_2*v_2 &= m'*v'\\
		&= (m_1+m_2)*v'\\
		\shortintertext{Energie:}
		E_{kin_{1}}+E_{kin_{2}} &= E'_{kin}\\
		\dfrac{1}{2}*m_1*{v_1}^2+\dfrac{1}{2}*m_2*{v_2}^2 &= \dfrac{1}{2}*m'*{v'}^2\\
		&= \dfrac{1}{2}*(m_1+m_2)*{v'}^2
	\end{align}
	Die kinetische Energie scheint bei den durchgeführten Versuchen nicht erhalten zu sein. Das lässt sich jedoch durch die, für die Verformung (und Erwärmung) der Knetmasse benötigte Energie erklären. Kurz gesagt: Ein Teil der kinetischen Energie wird nicht als kinetische Energie erhalten, sondern zu einer anderen Form von Energie umgewandelt. Deshalb:
	\begin{align}
		\intertext{Nicht:}
		E_{kin_{1}}+E_{kin_{2}} &= E'_{kin}\\
		\intertext{Sondern:}
		E_{kin_{1}}+E_{kin_{2}} &= E'_{kin}+U\\
		U &= \dfrac{1}{2}*\dfrac{m_1*m_2}{m_1+m_2}*(v_1-v_2)^2
	\end{align}
	\paragraph{Merke:}
	Die zentrale Formel für die Impulse beim vollständig inelastischen Stoß lautet:
	\begin{align}
		p_1 + p_2 &= p'\\
		m_1*v_1+m_2*v_2 &= m'*v'\\
		m_1*v_1+m_2*v_2 &= (m_1+m_2)*v'\\
		\shortintertext{bzw. nach $ v' $ aufgelöst:}
		v' &= \dfrac{m_1*v_1+m_2*v_2}{m_1+m_2}
	\end{align}
	\paragraph{Wichtig:}
	$ v $ ist eine Vektorgröße. In den meisten Aufgabenstellungen wird dabei die Bewegung in eine Richtung (z.B. in $ x $-Richtung) untersucht. Daher kann $ v_1 $, $ v_2 $ und $ v' $ auch negativ sein.
	
	\emph{Hier fehlen die Blätter dazwischen drinnen!!!}\\
	\emph{Z.B. der vollständig elastische Stoß fehlt komplett!!!}
	
	\subsection{Alltagsbeispiel: Autocrash}
	Kollidiert ein Auto mit einer Wand, so wird sein Impuls $ p $ vollständig abgebaut. Es gilt:
	\begin{align}
		\Delta p &= m*\Delta v\\
		&= F*\Delta t
	\end{align}
	Diese Formeln gelten unter der Annahme, dass die Kraft konstant ist, d.h. dass im gleichen Zeitfenster die gleiche Menge an Impuls abgebaut wird. Wird der Impuls jedoch nicht gleichmäßig abgebaut, so muss man die Impulsänderungen Zeitintervallen zuordnen und daraus die momentane Kraft berechnen.
	\begin{equation}
		F = \dfrac{\Delta p}{\Delta t}
	\end{equation}
	Lassen wir die Zeitintervalle ganz klein werden, dann folgt:
	\begin{equation}
		F(t) = \lim\limits_{\Delta t\to 0}\dfrac{\Delta p}{\Delta t} = p'(t)
	\end{equation}
	
	\section{Mechanik der Kreis- und Rotationsbewegung}
	\subsection{Die Kreisbewegung}
	\paragraph{Versuch:}
	% Bild/Skizze
	\subparagraph{Beobachtung:}
	Der Punkt $ P $ bewegt sich mit \underline{konstanter} Geschwindigkeit auf einer Kreisbahn. In diesem Fall spricht man von der \underline{gleichförmigen Kreisbewegung}.\\
	\paragraph{Größen zur Beschreibung der Bewegung:}
	\begin{description}
		\item[Radius] $ r $
		\item[Winkel] $ \alpha $ oder (s.u.) $ \varphi $
		\item[Bogenstrecke] $ b $
		\item[Kreisbahngeschwindigkeit] $ v $
		\item[Zeit] $ t $
		\item[Periodendauer] $ T $ (Zeit für eine volle Umkreisung)
		\item[Frequenz] $ f $
		\item[Kreisfrequenz] $ \omega $
		\item[Kraft] $ F $
		\item[Beschleunigung] $ a $
	\end{description}
	
	\subsection{Das Bogenmaß $\varphi$}
	Um beispielsweise den Kreisbogen $ b $ zu berechnen muss man bei gegebenem Winkel $ \alpha $ wie folgt rechnen:
	\begin{equation}
		b = \dfrac{\alpha}{360°}*2*\pi*r
	\end{equation}
	Jetzt: \underline{deffiniere} den Winkel $ \varphi $ so, dass der Term für $ b $ am Ende nicht mehr $ 360° $ und $ 2*\pi $ enthält:
	\begin{align}
		b &= \dfrac{\varphi}{2*\pi}*2*\pi*r\\
		\Rightarrow\hspace{0.5cm} b &= \varphi*r
	\end{align}
	Findet z.B. eine volle Umdrehung statt (in Grad wäre das $ 360° $), dann folgt für das Bogenmaß $ \varphi $:
	\begin{align}
		\varphi &= \dfrac{b}{r} = \dfrac{\dfrac{360°}{360°}*2*\pi*r}{r}\\
		\nonumber\\
		&= 2*\pi
	\end{align}
	Die Einheit des Bogenmaßes ist der Radiant und ist dimensionslos. Der Vorteil des Bogenmaßes wird in den kommenden Übungen ersichtlich.
	
	\subsection{Grundlegende wichtige Gleichungen für die Kreisbewegung}
	\begin{align}
		\shortintertext{Bogenmaß}
		\varphi &= \dfrac{b}{r}\\
		\nonumber\\
		\shortintertext{Umwandlung zwischen Grad und Radiant}
		\dfrac{\varphi}{2*\pi} &= \dfrac{\alpha}{360°} \quad \Longleftrightarrow \quad \alpha = 180°*\dfrac{\varphi}{\pi} \quad \Longleftrightarrow \quad \varphi = \pi*\dfrac{\alpha}{180°}\\
		\nonumber\\
		\shortintertext{Kreisbahngeschwindigkeit}
		v &= \dfrac{2*\pi*r}{T}\\
		\nonumber\\
		\shortintertext{Umlaufzeit bzw. Periodendauer}
		T &= \dfrac{2*\pi*r}{v}\\
		\nonumber\\
		\shortintertext{Frequenz bzw. Umdrehungen pro Zeiteinheit}
		f &= \dfrac{1}{T}\\
		\nonumber\\
		\shortintertext{Winkelgeschwindigkeit bzw. Winkel pro Zeiteinheit}
		\omega &= \dfrac{2*\pi}{T} = 2*\pi*f = \dfrac{v}{r}\\
		\nonumber\\
		\intertext{Gleichförmige Kreisbewegung:}
		\varphi(t) &= \omega*t
	\end{align}
	
	\subsection{Die Zentripetalkraft}
	\paragraph{Versuch:}
	Eine Kugel rollt zunächst geradlinig durch den Raum. Durch Stöße soll sie auf eine Kreisbahn gebracht werden.
	% Bild\Skizze
	Um die Änderung der Bewegungsrichtung herbeizuführen braucht es eine Impulsänderung $ \Delta \vec{p} $, die wiederum durch Einwirkung einer Kraft $ \vec{F} $ während einer Zeitspanne $ \Delta t $ erbracht wird. Die Kräfte wirken zu jedem Zeitpunkt auf den Mittelpunkt des Kreisbogens. Daher wird die für diese Bewegung verantwortliche Kraft \underline{Zentripetalkraft} $ F_{ZP} $ (lat. \textit{centrum} = Mitte, lat. \textit{petere} = streben) bezeichnet.
	\paragraph{Frage:}
	Wovon hängt die aufzubringende Kraft ab?
	\begin{itemize}
		\item Masse des Körpers $ m $
		\item Radius $ r $
		\item Winkelgeschwindigkeit $ \omega $
	\end{itemize}
	\paragraph{Ergebnis:}
	\begin{itemize}
		\item Je größer $ m $, desto größer $ F $
		\item Je größer $ r $, desto größer $ F $
		\item Je größer $ \omega $, desto größer $ F $
	\end{itemize}
	\paragraph{Vorbemerkung zu weiteren Aufgaben der ZP:}
	Beispiel Kavalierstart (Gaspedal bei Start voll durchgedrückt):\\
	Effekt: Reifen drehen durch.\\
	Ursache: Die Kraft, die vom Motor aus auf das Rad übertragen wird, ist viel größer als die Haftreibungskraft, die dafür sorgt, dass die Kraft von den Reifen auf die Straße übertragen wird und somit das Auto beschleunigt wird. sobald die Reifen durchdrehen, ist die Gleitreibungskraft die Kraft, die die Beschleunigung des Wagens bewerkstelligt. Da $ \mu_{Gleit} < \mu_{Haft} $ ist, ist die Beschleunigung des Wagens kleiner, als im Fall bei haftenden Reifen.
	
	\subsection{Inertialsysteme und Newton 1}
	\paragraph{Wir wissen:}
	2. Newtonsches Gesetz: $ F = m*a $. Ist dann Newton 1 (``wenn keine Kraft, dann keine Änderung des Bewegungszustandes (also Beschleunigung 0?)'') in Anbetracht von Newton 2 überflüssig?
	\paragraph{Newton 1 sagt mehr:} Es legt Bezugssysteme (d.h. das Koordinatensystem, in dem die Bewegung von Körpern beschrieben wird) fest, die mit der Newtonschen Mechanik beschrieben werden. Insbesondere werden dadurch Koordinatensysteme von der Newtonschen Mechanik ausgeschlossen, deren Ursprung sich im Raum beschleunigt.
	
	\subsection{Die Rotationsenergie und das Trägheitsmoment}
	\paragraph{Versuch:}
	Zwei Konservendosen gleicher Masse $ m $ aber verschiedener Inhaltszusammensetzung (flüssige Suppe vs. ``fester'' Eintopf) werden auf einem Hang gleichzeitig rollen gelassen.
	\subparagraph{Beobachtung:}
	Die Suppe kommt zuerst unten an.
	% Bild/Skizze
	\subparagraph{Erklärung:}
	Bei der Dose mit dem festen Inhalt muss der Inhalt mitgedreht werden. Hierfür ist Energie notwendig (die Rotationsenergie $ E_{rot} $) die dann eben nicht mehr für die lineare Bewegung (gekoppelt an die kinetische Energie $ E_{kin} $) zur Verfügung steht.\\\\
	D.h. näherungsweise beträgt die Gesamtenergie der Dosen:
	\begin{align}
		E_{ges, fest} &= E_{kin}+E_{rot}\\
		E_{ges, fl\textit{\"u}ssig} &= E_{kin}+\underbrace{E_{rot}}_{\substack{\approx \ 0J\text{, da}\\\text{Flüssigkeit} \\ \text{nicht mit-}\\\text{rotiert}}} = E_{kin}
	\end{align}
	Ziel des Kapitels ist es nun, einen eleganten Ausdruck zu finden, mit dem sich die Rotationsenergie berechnen lässt.
	% Bild/Skizze
	Die Rotationsenergie ist nichts anderes als die Summe der kinetischen Energien der einzelnen Massenteile des Körpers. Es gilt also bei $ n $ Massenteilen im Körper:
	\begin{align}
		E_{rot} &= E_{kin_1}+E_{kin_2}+E_{kin_3}+...+E_{kin_n}\\
		&= \dfrac{1}{2}*m_1*v_1^2+\dfrac{1}{2}*m_2*v_2^2+\dfrac{1}{2}*m_3*v_3^2+...+\dfrac{1}{2}*m_n*v_n^2
	\end{align}
	\paragraph{Problem:}
	Die Berechnung der Geschwindigkeiten erweist sich als zeitraubend, da, wenn sich die Dose auch nur etwas schneller dreht, die Geschwindigkeit für jedes Massenteil neu berechnet werden muss.
	\paragraph{Lösung:}
	Ersetze die Geschwindigkeitsterme durch die Winkelgeschwindigkeit $ \omega $:
	\begin{align}
		E_{rot} &= \dfrac{m_1*v_1^2}{2}+\dfrac{m_2*v_2^2}{2}+\dfrac{m_3*v_3^2}{2}+...+\dfrac{m_n*v_n^2}{2}\\
		\nonumber\\
		&= \dfrac{m_1*r_1^2*v_1^2}{2*r_1^2}+\dfrac{m_2*r_2^2*v_2^2}{2*r_2^2}+\dfrac{m_3*r_3^2*v_3^2}{2*r_3^2}+...+\dfrac{m_n*r_n^2*v_n^2}{2*r_n^2}\\
		\nonumber\\
		&= \dfrac{m_1*r_1^2}{2}*\dfrac{v_1^2}{r_1^2}+\dfrac{m_2*r_2^2}{2}*\dfrac{v_2^2}{r_2^2}+\dfrac{m_3*r_3^2}{2}*\dfrac{v_3^2}{r_3^2}+...+\dfrac{m_n*r_n^2}{2}*\dfrac{v_n^2}{r_n^2}\\
		\nonumber\\
		&= \dfrac{1}{2}*(m_1*r_1^2*\omega^2+m_2*r_2^2*\omega^2+m_3*r_3^2*\omega^2+...+m_n*r_n^2*\omega^2)\\
		&= \dfrac{1}{2}*\underbrace{(m_1*r_1^2+m_2*r_2^2+m_3*r_3^2+...+m_n*r_n^2)}*\omega^2\\
		&= \dfrac{1}{2}* \hspace{3.8cm} J \hspace{3.8cm}*\omega^2
	\end{align}
	$J$ ist das sogenannte Trägheitsmoment. Es hat die Einheit $ kg*m^2 $. Wenn in einer Aufgabenstellung die Masse und Radien gegeben sind, dann rechnet man $J$ nach der Formel
	\begin{equation}
		J = m_1*r_1^2+m_2*r_2^2+m_3*r_3^2+...+m_n*r_n^2
	\end{equation}
	aus. Ansonsten haben aufopferungsvolle Physiker und Mathematiker für bekannte geometrische Objekte (z.B. Zylinder) ihr Blut vergossen, um eine kompakte Formel bereitzustellen. Zum Beispiel:
	\begin{equation}
		J_{Zylinder} = \dfrac{1}{2}*m*r^2
	\end{equation}
	(hierbei ist angenommen, dass die Masse im Zylinder gleichförmig verteilt ist. Mit $ r $ ist der Radius an den äußeren Rand des Zylinders gemeint.)
	\paragraph{Daraus folgt:}
	\begin{itemize}
		\item Je größer $ r $, desto größer $ J $ \quad (bei gleichbleibendem $ m $)
		\item Je größer $ m $, desto größer $ J $ \quad (bei gleichbleibendem $ r $)
	\end{itemize}
	\paragraph{Zusammenfassung:}
	Die Rotationsenergie $ E_{rot} $ berechnet sich zu
	\begin{align}
		E_{rot} &= \dfrac{1}{2}*J*\omega^2
		\shortintertext{wobei J das sogenannte Trägheitmoment ist und sich entweder über}
		J &= m_1*r_1^2+m_2*r_2^2+m_3*r_3^2+...+m_n*r_n^2
		\shortintertext{oder über eine angegebene Formel berechnet, wie z.B. beim Zylinder:}
		J_{Zylinder} &= \dfrac{1}{2}*m*r^2
		\intertext{Rollt ein Körper über eine Oberfläche hinweg, dann hat er sowohl kinetische Energie $ E_{kin} $ aus der linearen Bewegung als auch Rotationsenergie $ E_{rot} $ aus der Rotation. Es gilt dann:}
		E_{Ges} &= E_{rot}+E_{kin}\\
		&= \frac{1}{2}*J*\omega^2+\frac{1}{2}*m*v^2
		\shortintertext{$ \omega $ ist mit $ v $ dann über den Rollradius $ r_{roll} $ verbunden:}
		\omega &= \dfrac{v}{r_{roll}}
	\end{align}
	
	\subsection{Der Drehimpuls $ L $}
	Aus der Analogietabelle haben wir hergeleitet, dass das Analogon zum Impuls $ p $ der linearen Bewegung
	\begin{equation}
		p = m*v
	\end{equation}
	der Drehimpuls $ L $ der Rotationsbewegung
	\begin{equation}
		L = J*\omega
	\end{equation}
	ist. Im folgenden Versuch soll eine weitere Eigenschaft und auch Analogie zwischen Impuls und Drehimpuls ergründet werden.
	\paragraph{Versuch:}
	% Bild/Skizze
	Wenn die Person während der Rotationsbewegung die Arme anzieht, dann wird das Trägheitsmoment $ J $ kleiner. Jedoch sah man, dass die Rotationsgeschwindigkeit $ \omega $ währenddessen anstieg.
	\paragraph{Folgerung:}
	Der Drehimpuls $ L $ ist wie der Impuls $ p $ eine Erhaltungsgröße.
	
	\paragraph{Der Fall einer Katze:}
	\begin{enumerate}
		\item Katze schaut nach unten. Gesamtdrehimpuls $ L_{Ges} = 0 $.
		\item Katze zieht Vorderbeine an, streckt Hinterbeine aus $ \to \ \ J_{vorne} $ klein, $ J_{hinten} $ groß.
		% Bild/Skizze
		\item Vorder- und Hinterkörper drehen sich in entgegengesetzte Richtungen (zur Erhaltung des Gesamtdrehimpulses $ L_{Ges} $). Da $ J_{vorne} < J_{hinten} $ folgt $ \omega_{vorne} > \omega_{hinten} $.
		% Bild/Skizze
		\item Katze streckt Vorderbeine aus, zieht Hinterbeine an $ \to \ \ J_{vorne} > J_{hinten} $
		\item Katze dreht sich. $ \omega_{hinten} > \omega_{vorne} $ (mit entgegengesetztem Vorzeichen).
		% Bild/Skizze
		\item Katze streckt Extremitäten aus $ \to $ Bremsweg $ s $ wird größer\\
		$ \to $ $ a $ wird kleiner $ \to $ belastende Kraft $ F $ wird kleiner.
	\end{enumerate}
	\paragraph{Die Richtung des Drehimpulses:}
	Betrachte sich drehendes Rad (Masse auf Radius):
	% Bild/Skizze
	Berechne den Drehimpuls $ L $ (nehme an, dass sich nur ein einziges Massenstück auf dem Radius befindet):
	\begin{align}
		L &= \quad J\hspace{0.34cm}*\hspace{0.34cm}\omega\\
		&= \overbrace{m*r^2}*\overbrace{\dfrac{v}{r}}\\
		&= m*r*v\\
		&= \underbrace{m*v}*r\\
		&= \hspace{0.34cm}p\hspace{0.15cm}*\hspace{0.15cm}r
	\end{align}
	Den Drehimpuls eines Massenteilchens kann man also auch beschreiben als das (Vektor-) Produkt aus dem Impuls $ p $ des Teilchens und dessen Radius. Exakte Darstellung für einen allgemeinen Fall:
	% Bild/Skizze
	Der Drehimpuls $ L $ ergibt sich also aus dem Produkt der Geschwindigkeit und dem Abstand des Drehzentrums $ M $ zur geradlinigen Bahn des sich bewegenden Körpers.\\
	Der Impuls $ p $ hat eine Richtung. Daher sollte der Drehimpuls $ L $ auch eine Richtung haben.
	\paragraph{Frage:}
	In welch Richtung zeigt der Drehimpuls $ L $?
	% Bild/Skizze
	Der Drehimpulsvektor $ \vec{L} $ steht dabei senkrecht zu $ \vec{r} $ und $ \vec{v} $ entlang der Rotationsachse (dadurch ändert er seine Richtung nicht im Gegensatz zu $ \vec{v} $).
	\paragraph{Merkhilfe 1:}
	Zeigen die Finger der rechten Hand den Verlauf eines Massenpunktes eines Rotationskörpers an, dann zeigt der Drehimpulsvektor entlang des Daumens.
	% Bild/Skizze
	\paragraph{Merkhilfe 2:}
	% Bild/Skizze
	\begin{description}
		\item[$ \vec{r} $] Rechtes Bein
		\item[$ \vec{v} $] linkes Bein
		\item[$ \vec{L} $] Körperrumpf
	\end{description}
	
\end{document}














